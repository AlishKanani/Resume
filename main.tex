\documentclass[]{resume}


\begin{document}

%%%%%%%%%%%%%%%%%%%%%%%%%%%%%%%%%%%%%%
%
%     COLUMN ONE
%
%%%%%%%%%%%%%%%%%%%%%%%%%%%%%%%%%%%%%%

\begin{minipage}[t]{0.33\textwidth} 
\vspace{-2em}
\begin{large}
	\headername{Alish Kanani}\\
\end{large}
Pre-final Year (B.Tech)\\
Electrical  Engineering\\ 
Indian Institute of Technology (IIT) Jodhpur  \\ 
%%%%%%%%%%%%%%%%%%%%%%%%%%%%%%%%%%%%%%
%     EDUCATION
%%%%%%%%%%%%%%%%%%%%%%%%%%%%%%%%%%%%%%
\vspace{-1em}
\section{Education}
\vspace{-0.5em}
\noindent\rule{5cm}{0.4pt}\\

\vspace{-0.7em}
\runsubsection{B.Tech - Electrical Engineering}\\
IIT Jodhpur \\
Expected 2021\\
CGPA : 8.75/10 (Till 5th Semester)\\

\vspace{-0.5em}
\runsubsection{Higher Secondary (HSC)}\\
Ashadeep Science Bhavan\\
2017 | Surat, Gujrat \\
Percentage: 94.4\%\\

\vspace{-0.5em}
\runsubsection{Secondary (SSC)}\\
M N J Patel High School\\
2015 | Surat, Gujrat \\
Percentage: 87.66\%\\
\sectionsep

\vspace{-1.5em}
%%%%%%%%%%%%%%%%%%%%%%%%%%%%%%%%%%%%%%
%     COURSEWORK
%%%%%%%%%%%%%%%%%%%%%%%%%%%%%%%%%%%%%%
\section{Relevant Courses}
\vspace{-0.5em}
\noindent\rule{5cm}{0.4pt}\\

\vspace{-0.7em}
\runsubsection{Electrical Core}\\
Digital Logic and Design\\
Microprocessors and Microcontrollers\\
Analog Electronics\\
Signal and Power Integrity*\\
Power Electronics\\
Circuit Theory\\
Communication Systems\\
Signals and Systems\\
Electrical Machine\\
Power System\\
Physics of Semiconductor Device\\
Digital Signal Processing*\\
Control Systems*\\
(* Ongoing courses)\\

\vspace{-0.5em}
\runsubsection{Fundamental Mathematics}\\
Real Analysis and Linear Algebra\\
Complex Analysis and Differential Equations\\
Probability and Statistics\\
\sectionsep

\vspace{-1.5em}
%%%%%%%%%%%%%%%%%%%%%%%%%%%%%%%%%%%%%%
%     SKILLS
%%%%%%%%%%%%%%%%%%%%%%%%%%%%%%%%%%%%%%
\section{Skills}
\vspace{-0.5em}
\noindent\rule{5cm}{0.4pt}

\vspace{0.5em}
\runsubsection{Language}\\
C\\
Python (Basic)\\
MATLAB\\
Perl(Basic)\\

\vspace{-0.5em}
\runsubsection{HDL}\\
Verilog\\

\vspace{-0.5em}
\runsubsection{Software}\\
Xilinx (ISE)\\
Synopsys DC\\
Cadence Virtuoso(Basic)\\
Simulink\\
Iverilog\\
Arduino\\
\sectionsep

%%%%%%%%%%%%%%%%%%%%%%%%%%%%%%%%%%%%%%
%
%     COLUMN TWO
%
%%%%%%%%%%%%%%%%%%%%%%%%%%%%%%%%%%%%%%

\end{minipage} 
\hfill
\begin{minipage}[t]{0.66\textwidth} 
% \descript{BS in Computer ence}
\vspace{-2em}
\hspace*{1pt}\hfill    \\
\hspace*{1pt}\hfill    \\
\hspace*{1pt}\hfill Mob. : +91 9664515665\\ 
\hspace*{1pt}\hfill Email. : \textbf{\href{mailto:kanani.1@iitj.ac.in}{\underline{kanani.1@iitj.ac.in}}} \\
\hspace*{1pt}\hfill Website : \textbf{\href{http://alishkanani.github.io}{\underline{alishkanani.github.io}}} \\
\hspace*{1pt}\hfill Github : \textbf{\href{https://github.com/AlishKanani}{\underline{github.com/AlishKanani}}}\\
\hspace*{1pt}\hfill 
LinkedIn : {\href{https://www.linkedin.com/in/alishkanani}{\underline{www.linkedin.com/in/alishkanani}}} \\

\vspace{-2em}
\section{Internship}
\vspace{-0.5em}
\noindent\rule{12.5cm}{0.4pt}

\vspace{0.2em}
\hspace{1em}
\runsubsection{Accuracy Configurable Arithmetic Circuit | IIT Ropar}\\
\vspace{-1em}\\
\hspace*{1em}Research Intern | May 2019 - July 2019 | Guide: \href{mailto: neeraj@iitrpr.ac.in }{\underline{Dr Neeraj Goel}}\\
\vspace*{-3em}\\
\descript{}
\begin{itemize}
    \item Studied various existing approximate binary adders and multipliers 
    \vspace{-0.6em}\\
    \item Proposed \custombold{An Accuracy Configurable Adder} and helped in \custombold{An Accuracy-Configurable Rounding-Based Multiplier}
    \vspace{-0.6em}\\
    \item Compared proposed algorithms with state of the art algorithms in Synopsys Design Compiler and in octave.
    % \vspace{-0.6em}
    % \item Submitted two papers in The IEEE International Symposium on\\ Circuits and Systems (ISCAS), 2020
\end{itemize}
\sectionsep

\vspace{-1.5em}
%%%%%%%%%%%%%%%%%%%%%%%%%%%%%%%%%%%%%%
%     PROJECTS
%%%%%%%%%%%%%%%%%%%%%%%%%%%%%%%%%%%%%%
\section{Projects}
\vspace{-0.5em}
\noindent\rule{12.5cm}{0.4pt}

\vspace{0.2em}
\hspace{1em}
\runsubsection{DHVANIK - Wearable Tympanometric Diagnostic Tool for Middle \hspace*{1.2em}Ear Ailments}\\
\hspace*{1em} Inter IIT Project | Sep 2019 - Dec 2019 | Guide: Dr Arpit Khandelwal\\ 
\vspace{-2em}
\descript{}
\begin{itemize}
    \item Developed a low-cost, user-friendly, IoT enabled, head-phone sized tympanometer that detects middle ear problem
    \vspace{-0.6em}\\
    \item Made business model for IICDC 2019 and Inter IIT 2019
\end{itemize}
\sectionsep

\vspace{-0.5em}
\hspace{1em}
\runsubsection{Optimisation of 32 bit adders}\\
\hspace*{1em} B.Tech Project | Jan 2019 - Apr 2019 | Guide: Dr S. P. Tiwari\\ 
\vspace{-2em}
\descript{}
\begin{itemize}
    \item Studied six different algorithms to add two binary numbers 
    \vspace{-0.6em}\\
    \item Compared delay, area and power of these algorithms in \custombold{Xilinx-ise}
    \vspace{-0.6em}\\
    \item Project report and code:  {\href{https://github.com/AlishKanani/32bitAdders}{\underline{github.com/AlishKanani/32bitAdders/}}} \\
\end{itemize}
\sectionsep

\vspace{-0.5em}
\hspace{1em}
\runsubsection{AES Data Encryption on FPGA}\\
\hspace*{1em} Gymkhana Project | Sep 2018 - May 2019   \\ 
\vspace{-2em}
\descript{}
\begin{itemize}
    \item The aim of the project was to develop an FPGA based encryption engine to facilitate massive data encryption and decryption.
    \vspace{-0.6em}\\
    \item Project link: {\href{https://github.com/AlishKanani/AES}{\underline{github.com/AlishKanani/AES/}}}
\end{itemize}
\sectionsep

\vspace{-0.5em}
\hspace{1em}
\runsubsection{NETRA- Indoor Navigator for Visually Impaired}\\
\hspace*{1em}Texas Instruments IICDC Competition | Aug 2018 - May 2019\\ 
\vspace{-2em}
\descript{}
\begin{itemize}
    \item Applied dead reckoning for indoor navigation without expensive infrastructure.
    \vspace{-0.6em}\\
    \item Implemented on \custombold{Beaglebone black} using 9 axis \custombold{IMU}
    \vspace{-0.6em}\\
    \item Audio and haptic feedback for navigation especially for the visually impaired
\end{itemize}
\sectionsep


\vspace{-1.5em}
%%%%%%%%%%%%%%%%%%%%%%%%%%%%%%%%%%%%%%
%     POR
%%%%%%%%%%%%%%%%%%%%%%%%%%%%%%%%%%%%%%

\section{Positions of Responsibility}
\vspace{-0.5em}
\noindent\rule{12.5cm}{0.4pt}

\vspace{0.2em}
\hspace{1em}
\runsubsection{CAPTAIN} Electronics Club | Aug 2018 - May 2019\\
\vspace{-2em}
\descript{}
\begin{itemize}
    \item Coordinated and managed year round activities and finance of the Electronics Club at IIT Jodhpur
\end{itemize}

\vspace{-0.4em}
\hspace{1em}%
\runsubsection{STUDENT GUIDE} Counselling Service | Aug 2018 - May 2019\\
\vspace{-2em}
\descript{}
\begin{itemize}
    \item Mentored freshmen students as a Student Guide for their smooth transitioning into college/hostel life.
\end{itemize}


\vspace{-3ex}
%%%%%%%%%%%%%%%%%%%%%%%%%%%%%%%%%%%%%%
%     Achievements
%%%%%%%%%%%%%%%%%%%%%%%%%%%%%%%%%%%%%%
\section{Achievements}
\vspace{-0.5em}
\noindent\rule{12.5cm}{0.4pt}

\vspace{-0.7em}
\begin{itemize}
  \item Accepted oral presentation on Approximate Multiplication at Research Conclave, IIT Guwahati.
  \vspace{-0.6em}
  \item Presented a poster on DHVANIK at Industry Day, IIT Jodhpur.  
  \vspace{-0.6em}
  \item Secured \custombold{Bronze} medal in Inter IIT Techmeet 2019 
  \vspace{-0.6em}
  \item Led the team of project NETRA and DHVANIK which reached the \custombold{semi-finals} of \custombold{DST} and \custombold{Texas Instruments} IICDC 2018 and 2019.
  \vspace{-0.6em}
  \item Won {\href{https://drive.google.com/file/d/1tH7FADsvsnGYCFPdl-HvmgBuhCWdn90d/view}{\underline{First Place in the Analog Designs}}} - an online contest by \custombold{Texas Instruments University Program.}
  \vspace{-0.6em}
  \item Placed among the top \custombold{0.5\% of 1.4 million} applicants in JEE Advanced 2017
\end{itemize}
\end{minipage} 
\end{document}  \documentclass[]{article}